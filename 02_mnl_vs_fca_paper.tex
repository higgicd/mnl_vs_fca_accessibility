\documentclass{article}

\usepackage{arxiv}

\usepackage[utf8]{inputenc} % allow utf-8 input
\usepackage[T1]{fontenc}    % use 8-bit T1 fonts
\usepackage{lmodern}        % https://github.com/rstudio/rticles/issues/343
\usepackage{hyperref}       % hyperlinks
\usepackage{url}            % simple URL typesetting
\usepackage{booktabs}       % professional-quality tables
\usepackage{amsfonts}       % blackboard math symbols
\usepackage{nicefrac}       % compact symbols for 1/2, etc.
\usepackage{microtype}      % microtypography
\usepackage{lipsum}
\usepackage{graphicx}

\title{Accessibility to Primary Care Physicians: Comparing Floating
Catchments with a Utility-based Approach}

\author{
  }


% Pandoc citation processing
\newlength{\csllabelwidth}
\setlength{\csllabelwidth}{3em}
\newlength{\cslhangindent}
\setlength{\cslhangindent}{1.5em}
% for Pandoc 2.8 to 2.10.1
\newenvironment{cslreferences}%
  {}%
  {\par}
% For Pandoc 2.11+
\newenvironment{CSLReferences}[2] % #1 hanging-ident, #2 entry spacing
 {% don't indent paragraphs
  \setlength{\parindent}{0pt}
  % turn on hanging indent if param 1 is 1
  \ifodd #1 \everypar{\setlength{\hangindent}{\cslhangindent}}\ignorespaces\fi
  % set entry spacing
  \ifnum #2 > 0
  \setlength{\parskip}{#2\baselineskip}
  \fi
 }%
 {}
\usepackage{calc} % for calculating minipage widths
\newcommand{\CSLBlock}[1]{#1\hfill\break}
\newcommand{\CSLLeftMargin}[1]{\parbox[t]{\csllabelwidth}{#1}}
\newcommand{\CSLRightInline}[1]{\parbox[t]{\linewidth - \csllabelwidth}{#1}\break}
\newcommand{\CSLIndent}[1]{\hspace{\cslhangindent}#1}



\begin{document}
\maketitle

\def\tightlist{}


\begin{abstract}
text goes here
\end{abstract}

\keywords{
    healthcare accessibility
   \and
    place-based accessibility
   \and
    utility-based accessibility
   \and
    destination choice model
  }

\hypertarget{introduction}{%
\section{Introduction}\label{introduction}}

The COVID-19 global pandemic has emphasized the importance of healthcare
accessibility, particularly access to primary care physicians, who
provide the first point of contact between patients and the healthcare
system. In Canada, the Canada Health Act states that all residents
should have ``reasonable access'' to healthcare. However, the 2017
Canadian Community Health Survey revealed that 15.3\% of Canadians aged
12 or over did not have a primary care physician, of whom 17.2\% stated
that there is no physician accessible within their area (StatsCan 2019).

Accessibility to healthcare services is defined by both spatial and
aspatial components (Joseph and Bantock 1982). Aspatial factors include
the cost and quality of healthcare services and the socioeconomic,
demographic, and mobility profile of potential users (Joseph and Bantock
1982). The second component considers geographic accessibility, which
can be defined as the potential to interact with a given set of
opportunities, such as healthcare facilities or primary care physicians,
from a given location using the transportation network (Hansen 1959).
Accessibility to healthcare can therefore be improved through either an
increase in the number of available opportunities or through
improvements to the transportation network.

In general, four approaches for calculating accessibility exist:
infrastructure-based approaches, which focus on the capacity of
transportation infrastructure; location-based approaches, which focus on
spatial distributions of opportunities; person-based approaches, which
focus on accessibility on an individual level; and utility-based
measures, which focus on the utility derived from interacting with the
opportunity or participating in an activity (Geurs and van Wee 2004). Of
these, place-based measures are the most common in the literature and,
of these, the family of ``floating catchment area'' (FCA) methods is one
of the most popular approaches for calculating place-based healthcare
accessibility. Because healthcare access is sensitive to demand and
supply, Luo and Wang (Luo and Wang 2003) (drawing on Radke and Mu
(2000)) introduced the Two-step Floating Catchment Area (2SFCA) method
that first estimates the demand for healthcare at service locations from
population zones and then allocates the level of service back to the
population zones using a binary measure of travel impedance.

Since then, various improvements have been made to the 2SFCA approach to
better capture the friction of distance. The original 2SFCA has been
criticized for over-estimating demand and under-estimating levels of
service in the estimation of accessibilities due to the
multiple-counting of populations that arises from the overlapping
catchments in a study area. In response, researchers have proposed
solutions such as the Three-step Floating Catchment Area (3SFCA) (Wan,
Zou, and Sternberg 2012), Modified 2SFCA (M2SFCA) (Delamater 2013), and
Balanced 2SFCA (B2SFCA) (Paez, Higgins, and Vivona 2019) methods. Of
these, the B2SFCA is the only approach that preserves the original
population and resulting levels of service in calculating floating
catchment accessibilities.

However, despite these innovations, FCA methods remain limited in
several ways. First, FCA approaches often inflate or deflate demand and
supply in the calculation of healthcare access. While the B2SFCA
remedies this, it does so by assigning fractions of populations to
clinics and service ratios to population zones. While the parameters of
the balanced method sum to the original zonal populations and
provider-to-population ratios, this fractional approach does not reflect
the ways in which individuals choose to visit facilities. Second, the
appeal of any given healthcare facility from the perspective of the
population is based solely on its distance or travel time from the
origin zone using the transportation network.

In response, this research utilizes a random utility-based formulation
for modelling accessibility to healthcare services. In contrast to FCA
approaches, each patient is, on average, assigned to a single clinic,
avoiding the issue of double-counting and inflation/deflation of the
demand and levels-of-service respectively in the 2SFCA methods and the
assignment of fractional individuals to clinics in the B2SFCA method.
Beyond travel time, this specification also allows the analyst to
include additional characteristics of the facilities that affect their
appeal, such as CONGESTION. To illustrate the potential of the MNL
approach, we compare it against the use of the 2SFCA and B2SFCA, both
using a continuous decay function.

\hypertarget{methodology}{%
\section{Methodology}\label{methodology}}

\hypertarget{floating-catchment-methods}{%
\subsection{Floating Catchment
Methods}\label{floating-catchment-methods}}

The 2-step floating catchment area (2SFCA) method, developed by Luo and
Wang, calculates accessibility to healthcare using catchment areas based
on a travel time threshold (Luo and Wang 2003). The first step of this
method is calculating the physician-to-population ratio, \(R_j\), for
each clinic at location \(j\):

\[
R_j = \frac{S_j}{\sum_i{P_iW_{ij}}}
\]

Where \(S_j\) is the number of physicians at clinic \(j\) and \(P_i\) is
the population of zone \(i\) weighted by some function of the travel
time \(W_{ij}\) between zones \(i\) and \(j\). In the original 2SFCA,
Luo and Wang (2003) utilize a binary impedance function:

\[
W_{ij} = f(t_{ij}) = \left\{
        \begin{array}{ll}
            1 & \quad t_{ij} \leq t_0 \\
            0 & \quad t_{ij} > t_0
        \end{array}
    \right.
\]

where the weight equals 1 for populations within the travel time
threshold \(t_0\) and zero beyond. In this case, Luo and Wang (2003) set
\(t_0 = 15\) minutes. The second step calculates accessibility \(A_i\)
for the population centres as the sum of the physician-to-population
ratios \(R_j\) weighted by the impedance function:

\[
A_i = \sum_j{R_jW_{ij}}
\]

While the 2SFCA approach is a special case of a gravity-based
accessibility measure, the binary impedance function used by Luo and
Wang (2003) does not consider the effects of competition and travel
impedance within a given catchment area. All clinics within a population
centre's catchment area are considered equally accessible, regardless of
distance, size, wait times, or any other measures of attractiveness.
Moreover, all clinics outside of a population centre's catchment area
are considered completely inaccessible. To remedy this, Luo and Qi
(2009) propose the Enhanced 2-step Floating Catchment Area (E2SFCA)
method that introduces categorical weights for different travel time
thresholds to account for travel impedance. Others have improved on the
2SFCA and E2SFCA by using variable catchment sizes (McGrail and
Humphreys 2009), continuous travel time decay functions (Dai 2010), and
adaptive approaches (Bauer and Groneberg 2016) to better reflect travel
time costs and the greater appeal of more proximate opportunities.

Researchers have also sought to improve the ways in which supply and
demand are modeled in floating catchment approaches. Previous research
has shown that both demand and supply can be inflated/deflated in FCA
methods (Delamater 2013; Paez, Higgins, and Vivona 2019; Wan, Zou, and
Sternberg 2012). This is a consequence of the overlapping floating
catchments that cause the populations in zones \(i\) to be counted
multiple times in the calculation of the provider-to-population ratio
\(R_j\). These levels-of-service are, in turn, counted multiple times
when allocated back to the population zones in the calculation of
\(A_i\). In response, Wan et al.~propose the use of additional Gaussian
weights to modify the binary impedance function used by Luo and Wang
(2003). Delamater's (2013) M2SFCA modifies the second step of the 2SFCA
approach by squaring the impedance function to increase the rate of
decay on the level of service. This is done to reflect the increased
friction population centres may experience when accessing healthcare
facilities in sub-optimally configured urban systems.

However, neither of these approaches fully resolves the issue of demand
and supply inflation/deflation. To that end, the B2SFCA approach from
Páez et al. (2019) that replaces the impedance functions with
row-standardized weights \(W_{ij}^{i}\) in the first step:

\[
R_j = \frac{S_j}{\sum_i{P_iW_{ij}^{i}}}
\] \[
W_{ij}^{i} = \frac{W_{ij}}{\sum_j W_{ij}}
\]

and with column-standardized weights \(W_{ij}^{j}\) in the second step:

\[
A_i = \sum_j{R_jW_{ij}^{j}}
\] \[
W_{ij}^{j} = \frac{W_{ij}}{\sum_i W_{ij}}
\]

In this formulation, the travel-time weighted populations sum to the
original population values and do not deflate the level-of-service at
the clinics. By extension, the levels of service available at the
population centres are not inflated through multiple counting. For this
research, we employ both the 2SFCA and B2SFCA approaches with a negative
exponential impedance function:

\[
W_{ij} = e^{-\beta t_{ij}}
\]

where \(\beta\) is a parameter that determines the decay of the function
and \(t_{ij}\) is the travel time between clinic \(j\) and population
centre \(i\). The \(\beta\) parameter is set to 0.05 as this is in the
range of typical auto travel time parameters in logit mode choice models
calibrated in the Greater Toronto and Hamilton Area. Travel times are
calculated based on car travel using a street network from OpenStreetMap
and the \texttt{r5r} routing tool (Pereira et al. 2021).

\hypertarget{utility-based-method}{%
\subsection{Utility-based Method}\label{utility-based-method}}

Despite offering balance across both stages of the FCA approach, the
B2SFCA results in fractional apportionment of the population and
levels-of-service between the population zones and clinics. To address
the limitations of existing methods, a novel methodology is developed
which assigns trips from population centres to clinics. The general form
of this function is as follows:

\[
T_{ij} = f(H_i, Z_j, R_j, t_{ij}, \beta)
\]

where:

\begin{itemize}
\item
  \(T_{ij}\) is the number of trips from zone \(i\) to clinic \(j\)
\item
  \(H_i\) is the number of households in zone \(i\)
\item
  \(Z_j\) is the number of doctors at clinic \(j\)
\item
  \(R_j\) is the demand-to-capacity ratio at clinic \(j\) (note this is
  inverted from the physician-to-population ratios used in previous FCA
  approaches)
\item
  \(t_{ij}\) is the travel time between zones \(i\) and \(j\), and
  \(\beta\) is a row vector of parameters to be estimated.
\end{itemize}

To estimate these parameters, information minimization is used as this
approach allows for the least-biased parameter estimation and has been
proven to be identical to utility maximization (Anas 1983). Based on
information minimization theory, the probability that a household in
zone i will visit clinic j can be estimated as follows:

\[
MAX_{T_{ij}} E = -\sum_{j \in J} \sum_{i \in I} T_{ij} log(T_{ij})
\]

Subject to the following constraints:

\[
\sum_{j \in J}T_{ij} = \alpha H_i \forall i \in I 
\] \[
\sum_{i \in I} \sum_{j \in J} T_{ij} t_{ij} = \bar{t}T 
\] \[
\sum_{i \in I} \sum_{j \in J} T_{ij} log(C_j) = \sum_{i \in I} \sum_{j \in J}T_{ij} log \omega Z_j = \bar{C}T 
\] \[
\sum_{i \in I} \sum_{j \in J} T_{ij} R_j = \bar{R}T
\]

where:

\begin{itemize}
\item
  \(I\) is the set of all residential zones
\item
  \(J\) is the set of all clinics
\item
  \(\alpha\) is the average number of visits to the doctor per household
\item
  \(\bar{t}\) is the average observed travel time for home-based trips
  to clinics
\item
  \(T\) is the total number of daily trips to clinics
\item
  \(C_j\) is the nominal service capacity at clinic \(j\)
\item
  \(\omega\) is the average number of patients served by a doctor per
  day
\item
  \(\bar{C}\) is the average observed nominal service capacity
\item
  \(\bar{R}\) is the average observed demand-to-capacity ratio
\item
  \(H\) is the total number of households
\item
  \(Z\) is the total number of primary care physicians
\end{itemize}

The service capacities and demand-to-capacity ratios are calculated as
follows:

\[
C_j = \omega Z_j
\] \[
R_j = \frac{\sum_{i \in I} T_{ij}}{C_j} = \frac{\sum_{i \in I} T_{ij}}{\omega Z_j}
\]

Solving this set of equations yields the following:

\[
T_{ij} = \alpha H_i P_{ij}
\]

This is a singly-constrained gravity model where the probability that a
household in zone \(i\) will visit clinic \(j\) is as follows:

\[
P_{ij} = \frac{e^{\beta_1 t_{ij} + \beta_{K+2} log \omega Z_j + \beta_{K + 3} R_j}}{\sum_j\prime e^{\beta_1 t_{ij}\prime + \beta_{K+2} log \omega Z_j\prime + \beta_{K + 3} R_j\prime}}
\]

Ideally, the \(\beta_1\), \(\beta_{K+2}\), and \(\beta_{K + 3}\)
parameters would be estimated iteratively in order to meet the outlined
constraints. However, due to a lack of observed data on trips to
doctors, these parameters are instead chosen based on the following
considerations:

\begin{itemize}
\item
  The \(\beta_1\) travel time impedance parameter is set to -0.05 based
  on previous choice models in the region and to align with the 2SFCA
  and B2SFCA approaches above
\item
  Random utility theory requires \(\beta_{K+2}\) to lie between 0 to 1
  in value. It is set equal to 1 in this case to maximize the
  attractiveness of larger clinics.
\item
  No theory is currently available to guide the choice of the
  \(\beta_{K+3}\) parameter and so -0.5 is chosen as a ``first guess''
  at a parameter value that would produce a reasonable sensitivity to
  clinic over-crowding, but not prevent over-crowding from occurring
\end{itemize}

These values ensure that increased travel times and demand-to-capacity
ratios reduce the probability that a household in zone \(i\) will visit
clinic \(j\), and increased capacity at clinic \(j\) increases the
probability.

In order to ensure that \(\bar{R}\) is approximately equal to 1, the
\(\alpha\) and \(\omega\) parameters are assumed to be 0.065 visits to
the doctor per household and 22 patients seen by a doctor per day, on
average, respectively. Since \(R_j\) is a function of \(T_{ij}\) and
vice-versa, an iterative approach is taken to estimate the \(R_j\)
values. The multonomial logit destination choice model ensures that
demand at clinics is not over-estimated, as each patient on average is
assigned to a single clinic and is not double counted, as occurs in the
2SFCA method. The end result is an approach that involves location
choice modelling by maximizing utility for patients, with clinics with
higher demand and longer travel times attracting fewer trips while
larger clinics and those closer to the origins attract more trips.

\hypertarget{utility-based-accessibility}{%
\subsection{Utility-based
Accessibility}\label{utility-based-accessibility}}

As shown by Anas (1983), multinomial logit models are equivalent to
gravity models. Following Ben-Akiva and Lerman (Ben-Akiva \& Lerman,
1985), accessibility can be defined within random utility theory as the
expected maximum utility for a trip. For the multinomial logit model, it
can be shown that this is the natural logarithm of the denominator of
the logit model (the so-called ``logsum'' or ``inclusive value'' term),
yielding for this model the following accessibility measure:

\[
a_i = log(\sum_{j\prime} e^{\beta_1 t_{ij}\prime + \beta_{K+2} log \omega Z_j\prime + \beta_{K + 3} R_j\prime})
\]

\hypertarget{study-area}{%
\section{Study Area}\label{study-area}}

The study area for this research is the City of Hamilton in Ontario,
Canada. Based on data from the 2016 Canadian Census of Population, the
population of Hamilton is 536,917 living in 211,596 households. Based on
the assumed \(\omega = 0.065\) visits to the doctor per household, this
results in 13,753.74 trips to the doctor entering the MNL model. The
left panel of Figure \ref{fig:study_area_map} plots population densities
in the Dissemination Area census small geographic units in the City of
Hamilton, highlighting that the higher-density urban core is surrounded
by lower-density suburbs that extend into land that is largely rural in
character.

Information on the count and location of primary care physicians was
obtained using the College of Physicians and Surgeons of Ontario's
online registration database. Clinic locations were geocoded based on
their address and records were aggregated to count the number of
physicians practicing at each unique location. The data for this paper
have been used previously in the paper by Páez et al. (2019), although
in this case we consider only clinics that are within the spatial extent
of the City of Hamilton. While this does introduce edge effects in the
calculation of accessibility, limiting the study extent to a closed
system permits calculation of the multinomial logit model's congestion
effects and utility-based accessibilities. In total, there are 631
primary care physicians available at clinics in the City of Hamilton in
our data. Note that this is not strictly the number of physicians, as
some physicians offer services at more than one clinic. Rather, it
reflects the availability of physicians at given locations. The right
panel of Figure \ref{fig:study_area_map} plots the location and total
number of available physicians at the clinic locations. This total
produces a city-wide average provider-to-population ratio of 117.52
primary care doctors available per 100,000 people. Based on our
assumption of \(\alpha = 22\) patients seen per doctor every day, this
results in a total capacity of 13,882 patient visits per day in the MNL
model formulation.

\begin{figure}
\includegraphics[width=1\linewidth]{./img/study_area_map} \caption{\label{fig:study_area_map}Population Density and Physician Locations}\label{fig:plot study_area_map}
\end{figure}

\hypertarget{results}{%
\section{Results}\label{results}}

\hypertarget{demand-and-clinic-level-of-service}{%
\subsection{Demand and Clinic Level of
Service}\label{demand-and-clinic-level-of-service}}

To discuss the results, we begin by focusing on the results associated
with how each of the methods calculates demand and levels of service at
the clinic locations. The level of service for the FCA approaches is the
local provider-to-population ratio for each clinic while the MNL model
calculates trip demand-to-patient capacity ratios. To make this
comparable, we first take the inverse of the MNL ratios to reflect
patient capacity-to-trip demand ratio (CDR). Figure \ref{fig:ppr_dist}
displays a pair plot of the density of each level-of-service statistic
and their relationship and correlations with one another. The plot
highlights how the 2SFCA and B2SFCA methods significantly differ in the
ways in which they allocate demand to the clinics. However, it is
interesting to note the relatively high correlation between the
population-to-provider ratios at the clinics in the 2SFCA and the
capacity-to-demand ratios ratios in the MNL model with the scatterplot
revealing some non-linearity in this relationship across the methods.

\begin{figure}
\includegraphics[width=1\linewidth]{./img/pair_plot_ppr} \caption{\label{fig:ppr_dist}Comparing Accessibility Distributions}\label{fig:ppr_dist_fig}
\end{figure}

Figure \ref{fig:los_maps} displays the levels of service for the clinic
locations. In general, more urban clinics tend to exhibit higher levels
of demand and lower levels of service across all three models. However,
the provider-to-population ratios for the individual clinics in the
2SFCA are extremely small compared to results from the B2SFCA model,
highlighting how the original method's multiple counting tends to
inflate the (travel time weighted) population numbers in each clinic's
catchment and deflate the level of service available at the clinics. In
contrast, the PPRs in the B2SFCA method are readily interpretable as the
local ratio of doctors per person for a given clinic considering the
(travel-time weighted and apportioned) populations within its catchment.
Similarly, the MNL CDRs reflect the relationship between the trip demand
and patient capacity based on the assumed rates. In terms of spatial
trends, results from the 2SFCA and MNL models suggest both calculate
higher levels of service at larger clinics in the urban core as well as
at a larger clinic in the city's rural north-west. In contrast, the
B2SFCA method generally produces higher levels of service in an
east-to-west direction. This could reflect boundary effects in the study
area that omit the large populations present in the rest of the Greater
Toronto Area on the northern side of Lake Ontario that may also have
access to these clinics by driving.

\begin{figure}
\includegraphics[width=1\linewidth]{./img/los_maps} \caption{\label{fig:los_maps}Comparing Accessibility Distributions}\label{fig:los_maps_fig}
\end{figure}

\hypertarget{healthcare-accessibility}{%
\subsection{Healthcare Accessibility}\label{healthcare-accessibility}}

With the levels of service calculated above, the three models then
calculate accessibility to healthcare services in Hamilton.
Distributions, relationships, and correlations for the accessibility
results are shown in Figure \ref{fig:access_dist}. In this case, all
three models are highly correlated. The 2SFCA and B2SFCA produce nearly
identical distributions of results, although in the case of the balanced
method, the accessibilities correspond to the sum of travel time
weighted and apportioned provider-to-population ratios available in the
population zones free of the inflation and deflation that occurs in the
2SFCA. In contrast, the scatterplots of the MNL results again highlight
some non-linearity in the way the utility-based accessibilities are
calculated compared to the FCA methods. The thinner tail of the MNL
distribution suggests the method also results in fewer population zones
with lower accessibility compared to the FCA methods.

\begin{figure}
\includegraphics[width=1\linewidth]{./img/pair_plot_access} \caption{\label{fig:access_dist}Comparing Accessibility Distributions}\label{fig:access_dist_fig}
\end{figure}

The general spatial trends are similar across all three models (Figure
\ref{fig:access_maps}). The absolute accessibility values differ in
accordance with the ways each method calculates its accessibility
results. The FCA methods define accessibility based on the
physician-to-population ratios of clinics, resulting in smaller values.
In contrast, the MNL method defines accessibilities as the logsum of the
multinomial logit model, resulting in larger values that have no direct
interpretation. In general, the highest accessibilities to primary care
physicians correspond to the downtown area of Hamilton, where a large
number of clinics are concentrated. Accessibility to physicians
generally decreases with increased distance from the downtown area.

\begin{figure}
\includegraphics[width=1\linewidth]{./img/access_maps} \caption{\label{fig:access_maps}Accessibility Results}\label{fig:plot access_maps}
\end{figure}

To better highlight significant differences in the spatial patterns of
accessibility produced by each method, Figure \ref{fig:access_diff_maps}
displays the absolute differences in the normalized accessibilities
across models. In general, the MNL method tends to produce higher
accessibilities for most zones compared to the FCA methods. In line with
the distributions above, the 2SFCA and B2SFCA models appear to be most
similar, with only slight absolute differences in the calculated
accessibility values.

\begin{figure}
\includegraphics[width=1\linewidth]{./img/access_diff_maps} \caption{\label{fig:access_diff_maps}Accessibility Differences}\label{fig:plot access_diff_maps}
\end{figure}

To examine whether there are any spatial patterns in these differences,
Figure \ref{fig:access_locm_maps} plots the results of Local Moran's I
tests. To make the values comparable, we first normalize each
accessibility vector between 0-1 and take the differences of the
normalized values across each approach. Next, the Local Moran's I is
calculated on the differences using queen-style contiguity weights, a
critical significance level of \(p=0.05\), and without correcting for
multiple testing. The resulting maps reveal some interesting patterns of
spatial clustering in the calculated normalized differences,
particularly across the two FCA models compared to the MNL model. Here,
differences in accessibility are greatest between the FCA and MNL
methods in the low-low (LL) cluster in the ring of outer suburbs that
surround the city. In contrast, the calculated accessibilities are more
consistent across the methods in the high-high (HH) cluster in the
central part of the city. Differences in the remaining zones are not
significant (NS).

This overall pattern is likely due to the way the MNL approach handles
clinic choices with populations tending to select their nearest clinics.
On the one hand, the greater accessibilities in more suburban and rural
zones likely derived from these populations accessing their closest
facility. On the other hand, this also means that fewer individuals from
more urban locations are competing for healthcare resources in these
more suburban and rural areas, leading to higher levels of service at
these suburban and rural clinics. In contrast, the FCA methods allocate
populations to all clinics within their catchment area using weights
derived from the impedance function. While this produces a smoothing of
the accessibilities, it can result in lower levels of service and
accessibility for clinics that populations may not actually use. This
effect seems to be minimized in more urban locations featuring higher
population densities and a greater number of clinics with available
physicians. Comparing the normalized results from the 2SFCA and the
B2SFCA models, the patterns of spatial clustering in the differences
appears to be less associated with the city's urban-suburban-rural urban
structure. While the B2SFCA method generally calculates slightly higher
accessibilities across much of the city, the methods are most dissimilar
in the south-west rural area.

\begin{figure}
\includegraphics[width=1\linewidth]{./img/access_locm_maps} \caption{\label{fig:access_locm_maps}Accessibility Difference Hot Spots}\label{fig:fig 4}
\end{figure}

\hypertarget{discussion-and-conclusions}{%
\section{Discussion and Conclusions}\label{discussion-and-conclusions}}

Since the 2SFCA was proposed by Luo and Wang (2003), the floating
catchment area approach has been a popular one for calculating
place-based accessibility to healthcare services that considers both the
supply and demand components, and several key innovations have been made
to FCA methods since. However, FCA methods are still limited in two
important ways. First, FCA methods do not fully consider aspects of
travel and choicemaking behaviour. Like many of the other place-based
accessibility measures, the only behavioural component of FCA methods is
the impedance function that is used to weight the value of opportunities
by the distance or travel time required to reach them. Second, FCA
approaches also tend to assign population demand and levels-of-service
to facilities or population zones in a fractional manner, using the
impedance function (and other adjustments) to weight each value within a
catchment area. Crucially, this use of overlapping catchment areas in
FCA approaches has been shown to bias results by inflating/deflating
supply and demand. While the B2SFCA approach proposed by Páez et al.
(Paez, Higgins, and Vivona 2019) rectifies this, it does so by
apportioning fractions of populations and levels-of-service through
adjustments to the impedance function.

To respond to these issues, this research developed a multinomial logit
destination choice model for calculating utility-based transportation
accessibility to primary care physicians. With its basis in random
utility theory, the MNL model considers several additional aspects that
define the appeal of clinics in addition to the travel time required to
reach them, including the number of physicians available at the clinic
and the level of crowding. The destination choice model also avoids
multiple-counting as the iterative fitting procedure results in the
assignment of each patient trip to a single clinic on average.

Comparisons of the MNL approach with 2SFCA and B2SFCA models using data
for the City of Hamilton suggests that the accessibility patterns
produced by each method are broadly similar, with the highest
accessibilities in the central core of the city where many clinics and
physicians are located. However, further analysis of the distributions,
correlations, and spatial clustering of accessibility differences
reveals that the MNL method produces generally higher accessibilities
throughout much of Hamilton with the greatest differences seen in the
ring of suburban and rural zones that surround the city. It seems likely
that these results arise from the MNL model assigning trips based on the
most proximate clinic for these residents while more urban residents are
being drawn to more urban clinics. In contrast, the FCA approaches
assign population values to all clinics within their catchment area and
all population zones share the levels-of-service of accessible clinics,
likely leading to higher demand and lower available supply at these
rural and suburban clinics.

For planning and policy, our analysis suggests that both the B2SFCA and
MNL approaches offer merit. While the 2SFCA is generally straightforward
to calculate with limited data requirements, it has been shown to return
biased results as a consequence of double counting that makes the
interpretation of provider-to-population ratios and accessibility scores
problematic. The B2SFCA, on the other hand, requires the same data as
the 2SFCA method but improves on it by preserving the population being
serviced and the level of service. Both the levels-of-service and
accessibilities calculated in the B2SFCA method are readily
interpretable as population-to-provider ratios. However, the only travel
behaviour component in both the 2SFCA and B2SFCA approaches is the
impedance function. While it does tend to result in the greatest weight
placed on the nearest locations in practice, particularly after the
adjustments made in the B2SFCA, it still results in a spreading or
smoothing of demand and supply. In contrast, the MNL model's
utility-based approach has a stronger behavioral foundation and
considers more aspects that define the appeal of particular clinics. It
also appears to produce what are arguably more realistic results in
suburban and rural areas. However, the MNL approach is more data-hungry
and required several parameter assumptions to be made on the part of the
research team. Further research can ascertain the sensitivity of the
results to these assumptions. Moreover, the accessibility scores have no
direct healthcare interpretation.

All methods in our comparative study are limited due to the imposition
of boundary effects that likely over-estimate levels-of-service at the
edges of the city and the consideration of only car travel. Moreover, we
only focus on the spatial component of accessibility and do not consider
the aspatial components that also play a significant role in defining an
individual's potential to reach and utilize healthcare services (Joseph
and Bantock 1982). In this regard, future research should utilize the
B2SFCA and MNL approaches for welfare analysis to measure place- and
utility-based accessibility to primary healthcare services for different
socioeconomic, demographic, and mobility profiles.

\hypertarget{author-contributions}{%
\section{Author Contributions}\label{author-contributions}}

The authors confirm contribution to the paper as follows: study
conception and design: X. Author, Y. Author; data collection: Y. Author;
analysis and interpretation of results: X. Author, Y. Author. Z. Author;
draft manuscript preparation: Y. Author. Z. Author. All authors reviewed
the results and approved the final version of the manuscript.

\hypertarget{references}{%
\section*{References}\label{references}}
\addcontentsline{toc}{section}{References}

\hypertarget{refs}{}
\begin{CSLReferences}{1}{0}
\leavevmode\hypertarget{ref-anas1983}{}%
Anas, Alex. 1983. {``Discrete Choice Theory, Information Theory and the
Multinomial Logit and Gravity Models.''} \emph{Transportation Research
Part B: Methodological} 17 (1): 13--23.
\url{https://doi.org/10.1016/0191-2615(83)90023-1}.

\leavevmode\hypertarget{ref-bauer2016}{}%
Bauer, Jan, and David A. Groneberg. 2016. {``Measuring Spatial
Accessibility of Health Care Providers {{}} Introduction of a Variable
Distance Decay Function Within the Floating Catchment Area (FCA)
Method.''} Edited by Kebede Deribe. \emph{PLOS ONE} 11 (7): e0159148.
\url{https://doi.org/10.1371/journal.pone.0159148}.

\leavevmode\hypertarget{ref-dai2010}{}%
Dai, Dajun. 2010. {``Black Residential Segregation, Disparities in
Spatial Access to Health Care Facilities, and Late-Stage Breast Cancer
Diagnosis in Metropolitan Detroit.''} \emph{Health \& Place} 16 (5):
1038--52. \url{https://doi.org/10.1016/j.healthplace.2010.06.012}.

\leavevmode\hypertarget{ref-delamater2013}{}%
Delamater, Paul L. 2013. {``Spatial Accessibility in Suboptimally
Configured Health Care Systems: A Modified Two-Step Floating Catchment
Area (M2sfca) Metric.''} \emph{Health \& Place} 24 (November): 30--43.
\url{https://doi.org/10.1016/j.healthplace.2013.07.012}.

\leavevmode\hypertarget{ref-geurs2004}{}%
Geurs, Karst T., and Bert van Wee. 2004. {``Accessibility Evaluation of
Land-Use and Transport Strategies: Review and Research Directions.''}
\emph{Journal of Transport Geography} 12 (2): 127--40.
\url{https://doi.org/10.1016/j.jtrangeo.2003.10.005}.

\leavevmode\hypertarget{ref-hansen1959}{}%
Hansen, Walter G. 1959. {``How Accessibility Shapes Land Use.''}
\emph{Journal of the American Institute of Planners} 25 (2): 73--76.
\url{https://doi.org/10.1080/01944365908978307}.

\leavevmode\hypertarget{ref-joseph1982}{}%
Joseph, Alun E., and Peter R. Bantock. 1982. {``Measuring Potential
Physical Accessibility to General Practitioners in Rural Areas: A Method
and Case Study.''} \emph{Social Science \& Medicine} 16 (1): 85--90.
\url{https://doi.org/10.1016/0277-9536(82)90428-2}.

\leavevmode\hypertarget{ref-luo2009}{}%
Luo, Wei, and Yi Qi. 2009. {``An Enhanced Two-Step Floating Catchment
Area (E2sfca) Method for Measuring Spatial Accessibility to Primary Care
Physicians.''} \emph{Health \& Place} 15 (4): 1100--1107.
\url{https://doi.org/10.1016/j.healthplace.2009.06.002}.

\leavevmode\hypertarget{ref-luo2003}{}%
Luo, Wei, and Fahui Wang. 2003. {``Measures of Spatial Accessibility to
Health Care in a GIS Environment: Synthesis and a Case Study in the
Chicago Region.''} \emph{Environment and Planning B: Planning and
Design} 30 (6): 865--84. \url{https://doi.org/10.1068/b29120}.

\leavevmode\hypertarget{ref-mcgrail2009}{}%
McGrail, Matthew R., and John S. Humphreys. 2009. {``Measuring Spatial
Accessibility to Primary Care in Rural Areas: Improving the
Effectiveness of the Two-Step Floating Catchment Area Method.''}
\emph{Applied Geography} 29 (4): 533--41.
\url{https://doi.org/10.1016/j.apgeog.2008.12.003}.

\leavevmode\hypertarget{ref-paez2019}{}%
Paez, Antonio, Christopher D. Higgins, and Salvatore F. Vivona. 2019.
{``Demand and Level of Service Inflation in Floating Catchment Area
(FCA) Methods.''} Edited by Tayyab Ikram Shah. \emph{PLOS ONE} 14 (6):
e0218773. \url{https://doi.org/10.1371/journal.pone.0218773}.

\leavevmode\hypertarget{ref-pereira2021}{}%
Pereira, Rafael H. M., Marcus Saraiva, Daniel Herszenhut, Carlos Kaue
Vieira Braga, and Matthew Wigginton Conway. 2021. {``R5r: Rapid
Realistic Routing on Multimodal Transport Networks with R5 in R.''}
\emph{Findings}, March. \url{https://doi.org/10.32866/001c.21262}.

\leavevmode\hypertarget{ref-radke2000}{}%
Radke, John, and Lan Mu. 2000. {``Spatial Decompositions, Modeling and
Mapping Service Regions to Predict Access to Social Programs.''}
\emph{Annals of GIS} 6 (2): 105--12.
\url{https://doi.org/10.1080/10824000009480538}.

\leavevmode\hypertarget{ref-statcan2019}{}%
StatsCan. 2019. {``Primary Health Care Providers, 2017.''} Statistics
Canada.

\leavevmode\hypertarget{ref-wan2012}{}%
Wan, Neng, Bin Zou, and Troy Sternberg. 2012. {``A Three-Step Floating
Catchment Area Method for Analyzing Spatial Access to Health
Services.''} \emph{International Journal of Geographical Information
Science} 26 (6): 1073--89.
\url{https://doi.org/10.1080/13658816.2011.624987}.

\end{CSLReferences}

\bibliographystyle{unsrtnat}
\bibliography{references.bib}


\end{document}
