\documentclass[]{elsarticle} %review=doublespace preprint=single 5p=2 column
%%% Begin My package additions %%%%%%%%%%%%%%%%%%%
\usepackage[hyphens]{url}

  \journal{Journal of Transport Geography} % Sets Journal name


\usepackage{lineno} % add
  \linenumbers % turns line numbering on
\providecommand{\tightlist}{%
  \setlength{\itemsep}{0pt}\setlength{\parskip}{0pt}}

\usepackage{graphicx}
%%%%%%%%%%%%%%%% end my additions to header

\usepackage[T1]{fontenc}
\usepackage{lmodern}
\usepackage{amssymb,amsmath}
\usepackage{ifxetex,ifluatex}
\usepackage{fixltx2e} % provides \textsubscript
% use upquote if available, for straight quotes in verbatim environments
\IfFileExists{upquote.sty}{\usepackage{upquote}}{}
\ifnum 0\ifxetex 1\fi\ifluatex 1\fi=0 % if pdftex
  \usepackage[utf8]{inputenc}
\else % if luatex or xelatex
  \usepackage{fontspec}
  \ifxetex
    \usepackage{xltxtra,xunicode}
  \fi
  \defaultfontfeatures{Mapping=tex-text,Scale=MatchLowercase}
  \newcommand{\euro}{€}
\fi
% use microtype if available
\IfFileExists{microtype.sty}{\usepackage{microtype}}{}
\bibliographystyle{elsarticle-harv}
\ifxetex
  \usepackage[setpagesize=false, % page size defined by xetex
              unicode=false, % unicode breaks when used with xetex
              xetex]{hyperref}
\else
  \usepackage[unicode=true]{hyperref}
\fi
\hypersetup{breaklinks=true,
            bookmarks=true,
            pdfauthor={},
            pdftitle={Accessibility to Primary Care Physicians: Comparing Floating Catchments with a Utility-based Approach},
            colorlinks=false,
            urlcolor=blue,
            linkcolor=magenta,
            pdfborder={0 0 0}}
\urlstyle{same}  % don't use monospace font for urls

\setcounter{secnumdepth}{0}
% Pandoc toggle for numbering sections (defaults to be off)
\setcounter{secnumdepth}{0}

% Pandoc citation processing
\newlength{\cslhangindent}
\setlength{\cslhangindent}{1.5em}
\newlength{\csllabelwidth}
\setlength{\csllabelwidth}{3em}
% for Pandoc 2.8 to 2.10.1
\newenvironment{cslreferences}%
  {}%
  {\par}
% For Pandoc 2.11+
\newenvironment{CSLReferences}[2] % #1 hanging-ident, #2 entry spacing
 {% don't indent paragraphs
  \setlength{\parindent}{0pt}
  % turn on hanging indent if param 1 is 1
  \ifodd #1 \everypar{\setlength{\hangindent}{\cslhangindent}}\ignorespaces\fi
  % set entry spacing
  \ifnum #2 > 0
  \setlength{\parskip}{#2\baselineskip}
  \fi
 }%
 {}
\usepackage{calc}
\newcommand{\CSLBlock}[1]{#1\hfill\break}
\newcommand{\CSLLeftMargin}[1]{\parbox[t]{\csllabelwidth}{#1}}
\newcommand{\CSLRightInline}[1]{\parbox[t]{\linewidth - \csllabelwidth}{#1}\break}
\newcommand{\CSLIndent}[1]{\hspace{\cslhangindent}#1}

% Pandoc header



\begin{document}
\begin{frontmatter}

  \title{Accessibility to Primary Care Physicians: Comparing Floating
Catchments with a Utility-based Approach}
    \author[Some University]{Author One}
   \ead{Some Email} 
    \author[Some University]{Author One\corref{1}}
   \ead{Some Email} 
    \author[Some University]{Author One}
   \ead{Some Email} 
    \author[Some University]{Author One}
   \ead{Some Email} 
      \address[Some University]{Address}
      \cortext[1]{Corresponding Author}
  
  \begin{abstract}
  Floating Catchment Area (FCA) methods are a popular choice for
  modelling accessibility to healthcare services because of their
  ability to consider both supply and demand. However, FCA methods do
  not fully consider aspects of travel and choicemaking behaviour as the
  only behavioural component is the impedance function. FCA approaches
  also tend to assign population demand to clinics and levels-of-service
  to population zones in an overlapping manner that has been shown to
  inflate/deflate supply and demand. While the adjustments proposed in
  the recent ``Balanced FCA'' method can rectify this, it apportions
  population and levels of service in a fractional manner. In response,
  this research proposes a utility-based measure of healthcare
  accessibility based on a multinomial logit (MNL) destination choice
  model that avoids the multiple-counting issue in FCA methods. It also
  considers additional behavioural aspects that define the appeal of
  clinics in addition to the travel time required to reach them,
  including their capacity and level of crowding. Comparisons of the MNL
  approach with the original and balanced FCA models using data for the
  City of Hamilton, Canada, suggests that while the accessibility
  patterns produced by each method are broadly similar, some key
  differences exist in the calculated accessibilities and their spatial
  patterns. The MNL model in particular estimates higher accessibilities
  in suburban and rural areas. After considering their strengths and
  weaknesses, we argue that both the FCA and MNL approaches offer merit
  for planning and policy.
  \end{abstract}
   \begin{keyword} healthcare accessibility place-based
accessibility utility-based accessibility destination choice
model accessibility analysis\end{keyword}
 \end{frontmatter}

\hypertarget{introduction}{%
\section{Introduction}\label{introduction}}

The global COVID-19 pandemic has emphasized the importance of healthcare
accessibility, particularly access to primary care physicians, who
provide the first point of contact between patients and the healthcare
system. In Canada, the Canada Health Act states that all residents
should have ``reasonable access'' to healthcare. However, the 2017
Canadian Community Health Survey revealed that 15.3\% of Canadians aged
12 or over did not have a primary care physician, of whom 17.2\% stated
that there is no physician accessible within their area (StatsCan,
2019).

Accessibility to healthcare services is defined by both spatial and
aspatial components (Joseph and Bantock, 1982). Aspatial factors include
the cost and quality of healthcare services and the socioeconomic,
demographic, and mobility profile of potential users. The second
component considers geographic accessibility, which can be defined as
the potential to interact with a given set of opportunities, such as
healthcare facilities or primary care physicians, from a given location
using the transportation network (Hansen, 1959). Accessibility to
healthcare can therefore be improved through either an increase in the
number of available opportunities or through improvements to the
transportation network.

In general, four approaches for calculating accessibility exist:
infrastructure-based, which focuses on the capacity of transportation
infrastructure; location-based, which focuses on spatial distributions
of opportunities; person-based, which focuses on accessibility on an
individual level; and utility-based, which focuses on the utility
derived from interacting with the opportunity or participating in an
activity (Geurs and van Wee, 2004). Place-based measures are the most
common in the literature and, of these, the family of ``floating
catchment area'' (FCA) methods is one of the most popular for
calculating measures of place-based healthcare accessibility that takes
the competition for opportunities into account. Because healthcare
access is sensitive to demand and supply, Luo and Wang (Luo and Wang,
2003) (drawing on Radke and Mu (2000)) introduced the Two-step Floating
Catchment Area (2SFCA) method that first estimates the demand for
healthcare at service locations from population zones and then allocates
the level of service back to the population zones using a binary measure
of travel impedance.

Since then, various improvements have been made to the 2SFCA approach
including adjustments to better capture the friction of distance
(Apparicio et al., 2017). The original 2SFCA has also been criticized
for over-estimating demand and under-estimating levels of service in the
estimation of accessibilities due to the multiple-counting of zonal
populations that arises from the overlapping catchments in a study area.
In response, researchers have proposed solutions such as the Three-step
Floating Catchment Area (3SFCA) (Wan et al., 2012), Modified 2SFCA
(M2SFCA) (Delamater, 2013), and Balanced 2SFCA (B2SFCA) (Paez et al.,
2019) methods. Of these, the B2SFCA is the only approach that preserves
the original population and resulting levels of service in calculating
floating catchment accessibilities.

However, despite these innovations, FCA methods remain limited in
several ways. First, FCA approaches often inflate or deflate demand and
supply in the calculation of healthcare access. While the B2SFCA
remedies this, it does so by assigning fractions of populations to
clinics and service ratios to population zones. Although the parameters
of the balanced method sum to the original zonal populations and
provider-to-population ratios, this fractional approach does not reflect
the ways in which individuals choose to visit facilities. Second, the
appeal of any given healthcare facility from the perspective of the
population is based solely on its distance or travel time from the
origin zone using the transportation network.

In response, this research utilizes a random utility-based formulation
for modelling accessibility to healthcare services. Compared to
place-based measures of accessibility, utility-based measures of access
have a solid grounding in travel behaviour theory (Geurs and van Wee,
2004; Miller, 2018) and allow the analyst to include any information
that corresponds to the expected value or attractiveness of travel
alternatives as well as characteristics of the individual or household
making the trip. While commonly used in alternatives appraisals for
transport infrastructure (de Jong et al., 2007), utility-based measures
of accessibility have not been as widely applied to capture other types
of access. However, they appear to be gaining some traction with recent
applications considering transit accessibility (Nassir et al., 2016),
first/last mile access to transit (Hasnine et al., 2019), regional
accessibility by income class (Jang and Lee, 2020), and accessibility to
parks (Macfarlane et al., 2020). To the best of our knowledge,
utility-based methods have not yet been applied to the problem of
healthcare access.

In response, this research proposes a utility-based measure of
healthcare accessibility based on a multinomial logit (MNL) destination
choice model. In contrast to FCA approaches, each patient is, on
average, assigned to a single clinic, avoiding the issue of
double-counting and inflation/deflation of the demand and
levels-of-service respectively in the 2SFCA methods and the assignment
of fractional individuals to clinics in the B2SFCA method. Beyond travel
time, this specification also allows the analyst to include additional
characteristics of the facilities that affect their appeal, such as
capacity and competition or crowding at the facility.

To illustrate the potential of the MNL approach, we compare it against
the use of the 2SFCA and B2SFCA, both using a continuous decay function.
To facilitate open and reproducible research in the spatial sciences
(Brunsdon and Comber, 2020; Páez, 2021), all data and code for this
analysis are contained within computational notebooks available at
(self-citation; .zip of files for review available anonymously via
Google Drive
\href{https://drive.google.com/file/d/1d66npiqIrawCU8DgNrYcztRJef80f7pd/view?usp=sharing}{link}).

\hypertarget{methodology}{%
\section{Methodology}\label{methodology}}

\hypertarget{floating-catchment-methods}{%
\subsection{Floating Catchment
Methods}\label{floating-catchment-methods}}

The 2SFCA method, developed by Luo and Wang (2003), calculates
accessibility to healthcare using catchment areas based on a travel time
threshold. The first step of this method is calculating the physician-
or provider-to-population ratio (PPR), \(R_j\), for each clinic at
location \(j\):

\begin{equation}
\label{eq:fca_rj}
R_j = \frac{S_j}{\sum_i{P_iW_{ij}}}
\end{equation}

\noindent where \(S_j\) is the number of physicians at clinic \(j\) and
\(P_i\) is the population of zone \(i\) weighted by some function of the
travel time \(W_{ij}\) between zones \(i\) and \(j\). In the original
2SFCA, Luo and Wang (2003) utilize a binary impedance function:

\begin{equation}
\label{eq:fca_W_ij_binary}
W_{ij} = f(t_{ij}) = \left\{
        \begin{array}{ll}
            1 & \quad t_{ij} \leq t_0 \\
            0 & \quad t_{ij} > t_0
        \end{array}
    \right.
\end{equation}

\noindent where the weight equals 1 for populations within the travel
time threshold \(t_0\) and zero beyond. In their paper, Luo and Wang
(2003) set \(t_0 = 15\) minutes. The second step calculates
accessibility \(A_i\) for the population centres as the sum of the
physician-to-population ratios \(R_j\) weighted by the impedance
function:

\begin{equation}
\label{eq:fca_A_i}
A_i = \sum_j{R_jW_{ij}}
\end{equation}

While the 2SFCA approach is a special case of a gravity-based
accessibility measure, the binary impedance function used by Luo and
Wang (2003) does not consider the effects of competition and travel
impedance within a given catchment area. All clinics within a population
centre's catchment area are considered equally accessible, regardless of
distance, size, wait times, or any other measures of attractiveness.
Moreover, all clinics outside of a population centre's catchment area
are considered completely inaccessible. To remedy this, Luo and Qi
(2009) propose the Enhanced 2-step Floating Catchment Area (E2SFCA)
method that introduces categorical weights for different travel time
thresholds to account for travel impedance. Others have improved on the
2SFCA and E2SFCA by using variable catchment sizes (McGrail and
Humphreys, 2009), continuous travel time decay functions (Dai, 2010),
and adaptive approaches (Bauer and Groneberg, 2016) to better reflect
travel time costs and the greater appeal of more proximate
opportunities.

Researchers have also sought to improve the ways in which supply and
demand are modeled in floating catchment approaches. Previous research
has shown that both demand and supply can be inflated/deflated in FCA
methods (Delamater, 2013; Paez et al., 2019; Wan et al., 2012). This is
a consequence of the overlapping floating catchments that cause the
populations in zones \(i\) to be counted multiple times in the
calculation of the provider-to-population ratio \(R_j\). These
levels-of-service are, in turn, counted multiple times when allocated
back to the population zones in the calculation of \(A_i\). In practice,
the inflation of demand in the first stage of the 2SFCA is generally
cancelled out in the second stage when calculating accessibility.
However, researchers may be interested in returning more meaningful
measures of levels-of-service at the clinics to support the allocation
of healthcare resources. In response, Wan et al. (2012) propose the use
of additional Gaussian weights to modify the binary impedance function
used by Luo and Wang (2003). This results in a steeper impedance
function that discounts demand and supply. Corrections have also been
made to discount accessibility. Delamater's (2013) M2SFCA modifies the
second step of the 2SFCA to increase the rate of decay on the
level-of-service available to population zones. This is done to address
the insensitivity of FCA approaches to the absolute distances required
to reach facilities in the calculation of accessibility. The result
reflects the increased friction population centres may experience when
accessing healthcare facilities in sub-optimally configured urban
systems.

However, neither of these approaches fully resolves the issue of demand
and supply inflation/deflation. To that end, the B2SFCA approach from
Páez et al. (2019) replaces the impedance functions with
row-standardized weights \(W_{ij}^{i}\) in the first step:

\begin{equation}
\label{eq:bfca_R_j}
R_j = \frac{S_j}{\sum_i{P_iW_{ij}^{i}}}
\end{equation}

\begin{equation}
\label{eq:bfca_W_ij_i}
W_{ij}^{i} = \frac{W_{ij}}{\sum_j W_{ij}}
\end{equation}

\noindent and with column-standardized weights \(W_{ij}^{j}\) in the
second step:

\begin{equation}
\label{eq:bfca_A_i}
A_i = \sum_j{R_jW_{ij}^{j}}
\end{equation}

\begin{equation}
\label{eq:bfca_W_ij_j}
W_{ij}^{j} = \frac{W_{ij}}{\sum_i W_{ij}}
\end{equation}

In this formulation, the travel-time weighted populations sum to the
original population values and do not deflate the level-of-service at
the clinics. These levels-of-service reflect local PPRs at the clinic
level. By extension, the levels-of-service available at the population
centres are not inflated through multiple counting. Nevertheless,
despite offering balance across both stages of the FCA approach, the
B2SFCA also results in fractional apportionment of the population and
levels-of-service between the population zones and clinics.

For this research, both the 2SFCA and B2SFCA approaches are specified
with a negative exponential impedance function:

\begin{equation}
\label{eq:W_ij}
W_{ij} = e^{-\beta t_{ij}}
\end{equation}

\noindent where \(\beta\) is a parameter that determines the decay of
the function and \(t_{ij}\) is the travel time between clinic \(j\) and
population centre \(i\). The \(\beta\) parameter is set to 0.05 as this
is in the range of typical auto travel time parameters in logit mode
choice models calibrated in the Greater Toronto and Hamilton Area
(Kasraian et al., 2020). Travel times are calculated based on car travel
using a street network from OpenStreetMap and the \texttt{r5r} routing
tool (Pereira et al., 2021).

\hypertarget{utility-based-method}{%
\subsection{Utility-based Method}\label{utility-based-method}}

To address the limitations of existing methods, a novel methodology for
deriving utility-based accessibility is developed which assigns trips
from households in population centres to clinics. The general form of
this function is as follows:

\begin{equation}
\label{eq:mnl_T_ij}
T_{ij} = f(H_i, Z_j, D_j, t_{ij}, \beta)
\end{equation}

\noindent where:

\begin{itemize}
\tightlist
\item
  \(T_{ij}\) is the number of trips from zone \(i\) to clinic \(j\)
\item
  \(H_i\) is the number of households in zone \(i\)
\item
  \(Z_j\) is the number of doctors at clinic \(j\)
\item
  \(D_j\) is the demand-to-capacity ratio at clinic \(j\) (note this is
  inverted from the physician-to-population ratios used in previous FCA
  approaches)
\item
  \(t_{ij}\) is the travel time between zones \(i\) and \(j\), and
  \(\beta\) is a row vector of parameters to be estimated.
\end{itemize}

To estimate these parameters, information minimization is used as this
approach allows for the least-biased parameter estimation and has been
proven to be identical to utility maximization (Anas, 1983). Based on
information minimization theory, the probability that a household in
zone \(i\) will visit clinic \(j\) can be estimated as follows:

\begin{equation}
\label{eq:mnl_MAX_T_ij}
MAX_{T_{ij}} E = -\sum_{j \in J} \sum_{i \in I} T_{ij} log(T_{ij})
\end{equation}

\noindent subject to the following constraints:

\begin{equation}
\label{eq:mnl_constraint1}
\sum_{j \in J}T_{ij} = \alpha H_i \forall i \in I
\end{equation}

\begin{equation}
\label{eq:mnl_constraint2}
\sum_{i \in I} \sum_{j \in J} T_{ij} t_{ij} = \bar{t}T
\end{equation}

\begin{equation}
\label{eq:mnl_constraint3}
\sum_{i \in I} \sum_{j \in J} T_{ij} log(C_j) = \sum_{i \in I} \sum_{j \in J}T_{ij} log \omega Z_j = \bar{C}T
\end{equation}

\begin{equation}
\label{eq:mnl_constraint4}
\sum_{i \in I} \sum_{j \in J} T_{ij} D_j = \bar{D}T
\end{equation}

\noindent where:

\begin{itemize}
\tightlist
\item
  \(I\) is the set of all residential zones
\item
  \(J\) is the set of all clinics
\item
  \(\alpha\) is the average number of visits to the doctor per household
\item
  \(\bar{t}\) is the average observed travel time for home-based trips
  to clinics
\item
  \(T\) is the total number of daily trips to clinics
\item
  \(C_j\) is the nominal service capacity at clinic \(j\)
\item
  \(\omega\) is the average number of patients served by a doctor per
  day
\item
  \(\bar{C}\) is the average observed nominal service capacity
\item
  \(\bar{D}\) is the average observed demand-to-capacity ratio
\item
  \(H\) is the total number of households
\item
  \(Z\) is the total number of primary care physicians
\end{itemize}

The service capacities and demand-to-capacity ratios are calculated as
follows:

\begin{equation}
\label{eq:mnl_C_j}
C_j = \omega Z_j
\end{equation}

\begin{equation}
\label{eq:mnl_D_j}
D_j = \frac{\sum_{i \in I} T_{ij}}{C_j} = \frac{\sum_{i \in I} T_{ij}}{\omega Z_j}
\end{equation}

\noindent Solving this set of equations yields the following:

\begin{equation}
\label{eq:mnl_T_ij_specific}
T_{ij} = \alpha H_i Pr_{ij}
\end{equation}

\noindent This is a singly-constrained gravity model where the
probability that a household in zone \(i\) will visit clinic \(j\) is as
follows:

\begin{equation}
\label{eq:mnl_Pr}
Pr_{ij} = \frac{e^{\beta_1 t_{ij} + \beta_{K+2} log \omega Z_j + \beta_{K + 3} D_j}}{\sum_j\prime e^{\beta_1 t_{ij}\prime + \beta_{K+2} log \omega Z_j\prime + \beta_{K + 3} D_j\prime}}
\end{equation}

Ideally, the three \(\beta\) parameters would be estimated iteratively
in order to meet the outlined constraints. However, due to a lack of
observed data on trips to doctors in the study area, these parameters
are instead chosen based on the following considerations:

\begin{itemize}
\tightlist
\item
  The \(\beta_1\) travel time impedance parameter is set to -0.05 to
  align with the choice model rationale outlined in the 2SFCA and B2SFCA
  approaches above.
\item
  Random utility theory requires the \(\beta_{K+2}\) capacity
  attractiveness parameter to lie between 0 to 1 in value. It is set
  equal to 1 in this case to maximize the attractiveness of larger
  clinics with more physicians.
\item
  No theory is currently available to guide the choice of the
  \(\beta_{K+3}\) parameter that influences sensitivity to overcrowding
  when trip demand exceeds the capacity to see patients at a clinic. In
  this case, -0.5 is chosen as a ``first guess'' value that would
  produce a reasonable sensitivity to clinic over-crowding, but not
  prevent over-crowding from occurring.
\end{itemize}

These values ensure that increased travel times and demand-to-capacity
ratios reduce the probability that a household in zone \(i\) will visit
clinic \(j\), while increased capacity at clinic \(j\) increases the
probability. Since \(D_j\) is a function of \(T_{ij}\) and vice-versa,
an iterative approach is taken to estimate the \(D_j\) values. The
multinomial logit destination choice model ensures that demand at
clinics is not over-estimated, as each patient on average is assigned to
a single clinic and is not double counted, as occurs in the 2SFCA
method. The end result is an approach that involves location choice
modelling by maximizing utility for patients, with clinics with higher
demand and longer travel times attracting fewer trips while larger
clinics that are uncongested and those closer to the origins attract
more trips.

\hypertarget{utility-based-accessibility}{%
\subsection{Utility-based
Accessibility}\label{utility-based-accessibility}}

While the probability of visiting a particular clinic is based on its
utility relative to the utility of others available within the choice
set, following Ben-Akiva and Lerman (1985), the expected maximum utility
from all destination choices available to a household can be understood
as a random utility theory-based measure of accessibility. For the
multinomial logit (MNL) model, it can be shown that this is the
logarithm of the denominator in Equation \ref{eq:mnl_Pr} (the so-called
``logsum'' or ``inclusive value'' term), yielding for this model the
following accessibility measure:

\begin{equation}
\label{eq:mnl_logsum}
A_i = log(\sum_{j\prime} e^{\beta_1 t_{ij}\prime + \beta_{K+2} log \omega Z_j\prime + \beta_{K + 3} D_j\prime})
\end{equation}

\noindent where accessibility is based not only on the utility of the
clinic with the greatest probability of visitation, but the utility of
all clinics available to a household considering travel impedance,
clinic size and capacity, trip-based demand for primary care physicians,
and congestion or crowding.

\hypertarget{study-area}{%
\section{Study Area}\label{study-area}}

The study area for this research is the City of Hamilton in Ontario,
Canada. Based on data from the 2016 Canadian Census of Population, the
population of Hamilton is 536,917 living in 211,596 households. The left
panel of Figure \ref{fig:study_area_map} plots population densities in
the Dissemination Areas (DAs) in the City of Hamilton, highlighting that
the higher-density urban core is surrounded by lower-density suburbs
that extend into land that is largely rural in character. DAs are the
smallest geographic unit for which socioeconomic and demographic census
data are publicly available.

Information on the count and location of primary care physicians was
obtained using the College of Physicians and Surgeons of Ontario's
online registration database. Clinic locations were geocoded based on
their address and records were aggregated to count the number of
physicians practicing at each unique location. The data for this paper
have been used previously by Páez et al. (2019), although in this case
we consider only clinics that are within the spatial extent of the City
of Hamilton. While this does introduce edge effects in the calculation
of accessibility, limiting the study extent to a closed system permits
calculation of the multinomial logit model's congestion effects and
utility-based accessibilities. In total, there are 631 primary care
physicians available at clinics in the City of Hamilton in our data.
Note that this is not strictly the number of physicians, as some
physicians offer services at more than one clinic. Rather, it reflects
the availability of physicians at given locations. The right panel of
Figure \ref{fig:study_area_map} plots the location and total number of
available physicians at the clinic locations. This total produces a
city-wide average physician-to-population ratio of 117.52 primary care
doctors available per 100,000 people.

Populating the MNL model requires the specification of several
parameters related to the study area. In order to ensure that the
average observed demand-to-capacity ratio \(\bar{D}\) is approximately
equal to 1, the \(\alpha\) and \(\omega\) parameters are assumed to be
22 patients seen by a doctor per day and 0.065 visits to the doctor per
household per day respectively. The patients per day number is derived
from the Canadian Institute for Health Information who reports that the
median number of patients seen by primary care physicians during a
typical work week in Ontario was 100 in 2019 (CIHI, 2020). At an assumed
20 patients per day over a 5-day work week and 50 weeks in a typical
year after holidays, this results in an estimated patient capacity of
approximately 3.2 million patients per year across the 631 primary care
physicians in the data, or 5.9 visits per person per year. On the other
hand, Vogel (2017) reports a Canada-wide average of 7.6 visits per
person per year in 2016, which would result in demand for approximately
4.1 million visits per year from the population residing in the City of
Hamilton.

Taking into consideration that Hamilton is part of the larger Greater
Golden Horseshoe region, meaning not all trips and visits are bounded by
the study area and that there are uncertainties surrounding the
estimated visits per year and physician practices, we slightly increased
the CIHI's number of patients seen per physician per week from 100 to
110. This corresponds to 22 visits per day and a capacity of 6.46 visits
per person per year. Based on our assumption of \(\alpha = 22\), this
results in a total capacity of 13,882 patient visits per day in the MNL
model formulation. Dividing the total estimated number of daily trips by
the number of households yields a household trip generation rate of
approximately 0.065 trips per household per day for 13,753.74 trips to
the doctor entering the MNL model.

\begin{figure}
\includegraphics[width=1\linewidth]{./img/study_area_map} \caption{\label{fig:study_area_map}Population Density and Physician Locations}\label{fig:fig 1 study_area_map}
\end{figure}

\hypertarget{results}{%
\section{Results}\label{results}}

\hypertarget{demand-and-clinic-level-of-service}{%
\subsection{Demand and Clinic Level of
Service}\label{demand-and-clinic-level-of-service}}

To compare the three methods, we focus first on the results associated
with how each of the methods calculates demand and levels of service at
the clinic locations. The level of service for the FCA approaches is the
local provider-to-population ratio (PPR) for each clinic while the MNL
model calculates trip demand-to-patient capacity ratios (DCR). To make
this comparable, we first take the inverse of the MNL ratios to reflect
patient capacity-to-trip demand ratio (CDR). Figure \ref{fig:ppr_dist}
displays a pair plot of the density of each level-of-service statistic
and their relationship and correlations with one another. The plot
highlights how the 2SFCA and B2SFCA methods are fundamentally similar in
the ways in which they allocate demand to the clinics with only a few
clinics above or below the scatterplot trend line. Likewise, it is
interesting to note the relatively high correlations between the PPRs at
the clinics in the FCA methods and the capacity-to-demand ratios in the
MNL model with the scatterplot revealing some non-linearity in this
relationship across the methods.

\begin{figure}
\includegraphics[width=1\linewidth]{./img/pair_plot_ppr} \caption{\label{fig:ppr_dist}Comparing Level of Service Distributions (Clinics)}\label{fig:fig 2 ppr_dist_fig}
\end{figure}

Figure \ref{fig:los_maps} displays the levels of service for the clinic
locations. In general, more urban clinics tend to exhibit higher levels
of demand and lower levels of service across all three models. However,
the PPR values for the individual clinics in the 2SFCA are extremely
small compared to results from the B2SFCA model, highlighting how the
original method's multiple counting tends to inflate the (travel time
weighted) population numbers in each clinic's catchment and deflate the
level of service available at the clinics. In contrast, the PPRs in the
B2SFCA method are readily interpretable as the local ratio of doctors
per person for a given clinic considering the (travel-time weighted and
apportioned) populations within its catchment. Similarly, the MNL CDRs
reflect the relationship between trip demand and patient capacity based
on the assumed rates. In terms of spatial trends, results from the 2SFCA
and MNL models suggest both calculate higher levels of service at larger
clinics in the urban core as well as at a larger clinic in the city's
rural north-west. In contrast, the B2SFCA method generally produces
higher levels of service in an east-to-west direction. This could
reflect boundary effects in the study area that omit the large
populations present in the rest of the Greater Toronto Area on the
northern side of Lake Ontario that may also have access to these clinics
by driving.

\begin{figure}
\includegraphics[width=1\linewidth]{./img/los_maps} \caption{\label{fig:los_maps}Mapping Levels of Service (Clinics)}\label{fig:fig 3 los_maps_fig}
\end{figure}

\hypertarget{healthcare-accessibility}{%
\subsection{Healthcare Accessibility}\label{healthcare-accessibility}}

With the levels of service calculated above, the three models then
calculate accessibility to healthcare services in Hamilton.
Distributions, relationships, and correlations for the accessibility
results are shown in Figure \ref{fig:access_dist}. In this case, all
three models are highly correlated. The 2SFCA and B2SFCA produce nearly
identical distributions of results. As above, their main difference is
in the scaling of parameters. In the 2SFCA, accessibilities can be
interpreted relative to the city-wide average provider-to-population
ratio of 0.0012 doctors per person. In the case of the balanced method,
the accessibilities correspond to the sum of travel time weighted and
apportioned provider-to-population ratios available in the population
zones free of the inflation and deflation that occurs in the 2SFCA. In
contrast, the scatterplots of the MNL results again highlight some
non-linearity in the way the utility-based accessibilities are
calculated compared to the FCA methods. The thinner tail of the MNL
distribution suggests the method also results in fewer population zones
with lower accessibility compared to the FCA methods.

\begin{figure}
\includegraphics[width=1\linewidth]{./img/pair_plot_access} \caption{\label{fig:access_dist}Comparing Accessibility Distributions (DAs)}\label{fig:fig 4 access_dist_fig}
\end{figure}

The general spatial trends are similar across all three models (Figure
\ref{fig:access_maps}). The absolute accessibility values differ in
accordance with the ways each method calculates its accessibility
results. The FCA methods define accessibility based on the
physician-to-population ratios of clinics, resulting in smaller values.
In contrast, the MNL method defines accessibilities as the logsum of the
denominator of the multinomial logit model, resulting in larger values
that have no direct physical interpretation. In general, the highest
accessibilities to primary care physicians correspond to the downtown
area of Hamilton, where a large number of clinics are concentrated.
Accessibility to physicians generally decreases with increased distance
from the downtown area.

\begin{figure}
\includegraphics[width=1\linewidth]{./img/access_maps} \caption{\label{fig:access_maps}Mapping Accessibility Results (DAs)}\label{fig:fig 5 access_maps}
\end{figure}

To better highlight significant differences in the spatial patterns of
accessibility produced by each method, Figure \ref{fig:access_diff_maps}
displays the absolute differences in normalized accessibilities across
models. To make the values comparable, we first normalize each
accessibility vector between 0-1 and take the differences of the
normalized values across each approach. In general, the MNL method tends
to produce higher accessibilities for most zones compared to the FCA
methods. In line with the distributions above, the 2SFCA and B2SFCA
models appear to be most similar, with only slight absolute differences
in the calculated accessibility values. Nevertheless, paired Wilcoxon
Rank Sum tests across all three types of normalized accessibilities are
statistically significant at \(p=0.05\), suggesting that while the
accessibility results are highly correlated, their distributions are
significantly different.

\begin{figure}
\includegraphics[width=1\linewidth]{./img/access_diff_maps} \caption{\label{fig:access_diff_maps}Normalized Accessibility Differences}\label{fig:fig 6 access_diff_maps}
\end{figure}

To examine whether there are any spatial patterns in these differences,
Figure \ref{fig:access_locm_maps} plots the results of Local Moran's I
tests. The Local Moran's I is calculated on the differences using
queen-style contiguity weights, a critical significance level of
\(p=0.05\), and without correcting for multiple testing. The resulting
maps reveal some interesting patterns of spatial clustering in the
calculated normalized differences, particularly across the two FCA
models compared to the MNL model. Here, differences in accessibility are
greatest between the FCA and MNL methods in the low-low (LL) cluster in
the ring of outer suburbs that surround the city where the MNL model
tends to estimate higher accessibilities. In contrast, the calculated
accessibilities are more consistent across the methods in the high-high
(HH) cluster in the central part of the city. Differences in the
remaining zones are generally not significant (NS) aside from a very
small number of high-low (HL) and low-high (LH) outliers.

This overall pattern is likely due to the way the MNL approach handles
clinic choices and accessibilities. Higher accessibilities in the urban
core are generally due to populations having access to a larger number
of clinics that are closer, have higher capacities, and are less
congested. The iterative trip-based calculations in the MNL model should
also result in fewer individuals from more urban locations competing for
healthcare resources at more suburban and rural clinics, leading to
higher accessibilities in these areas relative to the other methods. In
contrast, the FCA methods allocate populations to all clinics within
their catchment area using weights derived from the impedance function.
While this produces a smoothing of the accessibilities, it can result in
lower levels of service and accessibility for clinics that populations
may not actually use. This effect seems to be minimized in more urban
locations featuring higher population densities and a greater number of
clinics with available physicians. Comparing the normalized results from
the 2SFCA and the B2SFCA models, the patterns of spatial clustering in
the differences appears to be less associated with the city's
urban-suburban-rural urban structure. While the B2SFCA method generally
calculates slightly higher accessibilities across the western half of
the city, the 2SFCA produces slightly higher accessibilities in the
east.

\begin{figure}
\includegraphics[width=1\linewidth]{./img/access_locm_maps} \caption{\label{fig:access_locm_maps}Accessibility Difference Hot Spots}\label{fig:fig 7 access_locm_maps}
\end{figure}

\hypertarget{mnl-model-sensitivity-analysis}{%
\subsection{MNL Model Sensitivity
Analysis}\label{mnl-model-sensitivity-analysis}}

In order to assess the impact of the asserted \(\beta_{K+2}\) and
\(\beta_{K+3}\) parameters on the results generated by the MNL method, a
sensitivity analysis was undertaken. The \(\beta_{K+2}\) parameter that
influences the attractiveness of higher-capacity clinics was gradually
increased from 0.1 to 1 and the \(\beta_{K + 3}\) parameter that
influences sensitivity to congestion or crowding at the clinics was
gradually increased from -1 to -0.1. Increments of 0.1 are used for each
variable. Results are summarized by calculating average CDRs and
accessibilities across the clinics and DAs respectively in each of the
100 scenarios (Figure \ref{fig:sensitivity_plot}). Two example scenarios
are also created for illustration. In Scenario 1, the \(\beta_{K+2}\)
parameter was decreased from 1 to 0.5 relative to the original
calculations while the \(\beta_{K + 3}\) parameter was decreased from
-0.5 to -1 in Scenario 2 (increasing the sensitivity to overcrowding).

For the CDRs in the left panel of Figure \ref{fig:sensitivity_plot}, the
sensitivity analysis reveals that decreasing sensitivity to the
attractiveness of capacity (as \(\beta_{K+2}\) approaches 0.1) and
decreasing sensitivity to overcrowding (as \(\beta_{K+3}\) approaches
-0.1) combine to produce more balance between the supply of physician
capacities and patient demand, on average. Examining the clinic data in
greater detail, this weighting results in more trips being made to
smaller and more congested clinics relative to larger ones where there
is more supply relative to demand. Greater weight placed on facility
capacity (as \(\beta_{K+2}\) approaches 1) and high sensitivity to
overcrowding (as \(\beta_{K+3}\) approaches -1) also results in more
balanced CDRs, but in this case, more trips are made to larger clinics
that become more congested versus smaller ones that are less congested.
Scenarios along the diagonal exhibit relatively less balance, on
average, across the clinics.

In terms of accessibilities, average accessibilities are, in general,
more sensitive to changes in the \(\beta_{K+2}\) parameter than
\(\beta_{K+3}\). Comparing the original results against Scenario 1,
average accessibilities increase by around 22\% when \(\beta_{K+2}\)
increases from 0.5 to 1. In contrast, accessibilities in the original
scenario are about 11\% greater than those calculated from Scenario 2
where the \(\beta_{K+3}\) sensitivity to overcrowding parameter
increases in weight from -0.5 to -1. The greatest average
accessibilities in Figure \ref{fig:sensitivity_plot} result from high
attractiveness to clinic capacity and low weight on overcrowding
(\(\beta_{K+2} = 1\) and \(\beta_{K+3} = -0.1\)). This produces high
levels of average access as households benefit from the overall
availability of large clinics accessible by car despite the greater
levels of congestion that occur at smaller clinics.

To examine whether the sensitivity analysis impacts the spatial
distributions of calculated accessibilities, Figure
\ref{fig:sensitivity_maps} plots normalized accessibility results from
the original and two sensitivity scenarios. Although both adjustments to
the parameters result in decreased absolute accessibilities in Figure
\ref{fig:sensitivity_plot}, comparisons of normalized values suggest
there are no distinct spatial trends associated with changes in
\(\beta_{K + 2}\) and \(\beta_{K + 3}\) across the sensitivity
scenarios. Overall, both Figures \ref{fig:sensitivity_plot} and
\ref{fig:sensitivity_maps} indicate that the MNL results are relatively
robust with respect to the parameter value assumptions. CDR average
values are relatively constant across wide combinations of parameter
values (except at the extremes of values), and the relative spatial
distributions of average accessibilities are quite consistent as
parameter values change. In addition, both the average CDRs and
accessibility values change in expected ways as the parameters are
varied, providing some indication of behavioral soundness of the MNL
specification.

\begin{figure}
\includegraphics[width=1\linewidth]{./img/sensitivity_plot} \caption{\label{fig:sensitivity_plot}MNL Sensitivity Analysis Results}\label{fig:fig 8 sensitivity_tiles}
\end{figure}

\begin{figure}
\includegraphics[width=1\linewidth]{./img/sensitivity_maps} \caption{\label{fig:sensitivity_maps}Sensitivity Scenario Maps}\label{fig:fig 9 sensitivity_maps}
\end{figure}

\hypertarget{discussion-and-conclusions}{%
\section{Discussion and Conclusions}\label{discussion-and-conclusions}}

Since the 2SFCA was proposed by Luo and Wang (2003), the floating
catchment area approach has been a popular one for calculating
place-based accessibility to healthcare services that considers both the
supply and demand components and several key innovations have been made
to FCA methods since. However, FCA methods are still limited in two
important ways. First, FCA methods do not fully consider aspects of
travel and choicemaking behaviour. Like many of the other place-based
accessibility measures, the only behavioural component of FCA methods is
the impedance function that is used to weight the value of opportunities
by the distance or travel time required to reach them. Second, FCA
approaches also tend to assign population demand and levels-of-service
to facilities or population zones in an overlapping manner, using the
impedance function (and other adjustments) to weight each value within a
catchment area. Crucially, this use of overlapping catchment areas in
previous FCA approaches has been shown to inflate/deflate supply and
demand. Although the B2SFCA proposed by Páez et al. (2019) rectifies
this, it does so by apportioning fractions of populations and
levels-of-service through adjustments to the impedance function.

To respond to these issues, this research developed a multinomial logit
destination choice model for calculating utility-based transportation
accessibility to primary care physicians. While FCA approaches consider
accessibility in terms of provider-to-population ratios weighted by
distance or travel time, the MNL approach re-frames the measurement of
health accessibility into individual trips to visit primary care
physicians and the utility-bearing aspects of clinics. With its
comparatively strong basis in random utility theory, the MNL model
considers several additional aspects that define the appeal of clinics
in addition to the travel time required to reach them, including the
number of physicians available at the clinic and the level of crowding.
The destination choice model also avoids multiple-counting as the
iterative fitting procedure results in the assignment of each patient
trip to a single clinic on average.

Comparisons of the MNL approach with 2SFCA and B2SFCA models using data
for the City of Hamilton suggests that the accessibility patterns
produced by each method are broadly similar, with the highest
accessibilities in the central core of the city where many clinics and
physicians are located. However, further analysis of the distributions,
correlations, and spatial clustering of accessibility differences
reveals that the MNL method produces generally higher accessibilities
throughout much of Hamilton with the greatest differences seen in the
ring of suburban and rural zones that surround the city. These results
are generally a product of the MNL model assigning higher trip
probabilities to the most proximate clinic for suburban and rural
residents while more urban residents are being drawn to more urban
clinics. From this, the expected maximum utility (logsum) measure being
used to define accessibility recognizes that in addition to the benefits
derived from having access to the closest or largest clinics, people
will tend to find the ``best'' clinic for their needs. In contrast, the
FCA approaches assign population values to all clinics within their
catchment area and all population zones share the levels-of-service of
accessible clinics, likely leading to higher demand and lower available
supply at these rural and suburban clinics.

Our analysis suggests that both the FCA and MNL approaches offer merit.
The 2SFCA is generally straightforward to calculate with limited data
requirements and returns accessibilities that can be interpreted
relative to city-wide provider-to-population ratios. However, as a
consequence of multiple counting, the 2SFCA method has been shown to
significantly inflate population levels in the accessibility analysis
that deflate clinic levels of service. The B2SFCA requires the same data
as the 2SFCA method but preserves the population being serviced and
returns interpretable levels of service for clinics. Accessibilities
calculated using the B2SFCA method are interpretable as each population
centre's share of the levels-of-service available at clinics within
their catchment. Results from both the 2SFCA and B2SFCA approaches are
very highly correlated, indicating that while they return different
values, they fundamentally capture much of the same information.
However, the only travel behaviour component in both the 2SFCA and
B2SFCA approaches is the impedance function. While it does tend to
result in the greatest weight placed on the nearest locations in
practice, it still results in a spreading or smoothing of demand and
supply as populations are allocated across multiple clinics.

In contrast, the MNL model's utility-based approach has a stronger
behavioral foundation in terms of tripmaking and competition and
considers more aspects that define the appeal of particular clinics,
such as capacity. It also appears to produce what are arguably more
realistic results in suburban and rural areas, suggesting that compared
to the fixed travel time catchments used in the FCA approaches in the
present work, the MNL method can more dynamically account for changes in
supply and demand associated with variations in a city's urban
structure. The MNL approach, however, requires additional parameters to
be estimated relative to the other models. Calibrating the model's
parameters to match observed trip distributions would be ideal. Such
data are not available in the present case. There is no reason, however,
why the needed data could not be gathered, either as part of typical
large-scale travel surveys that are routinely collected in most urban
regions, or through custom surveys of clinic patients conducted by
public health agencies. The data could also be gleaned from government
health billing records, which are already often used in a wide variety
of epidemiological studies.

While the MNL expected maximum utilities associated with accessible
clinic destinations do not have the same direct healthcare
interpretation of the PPRs generated by the FCA approaches, they do have
an important economic interpretation They represent the consumer surplus
associated with clinic visits, and hence are a direct economic measure
of social welfare (Ben-Akiva and Lerman, 1985). Further, while not
undertaken in this paper, the logsum values can be converted into
equivalent units of either travel time or money, thereby providing a
more interpretable measurement, and opening the possibility of including
the measure in broader assessments of the benefits of investing in
greater heath care accessibility (Miller, 2018).

In terms of extensions to planning and policy, it has been argued that
compared to place-based measures, utility-based measures of
accessibility are more difficult to explain and understand in general
(Geurs and van Wee, 2004). Nevertheless, while FCA-based approaches can
account for some degree of competition when calculating accessibility
and may be easier to interpret, the results generated from such
place-based measures more disconnected from microeconomic theories
associated with choicemaking behaviour and consumer surplus that ground
utility-based models of access. Thus, while this paper focuses on
comparing and contrasting the two approaches, there is no reason why
there needs to be an ``either/or'' choice between the two methods. They
might most properly be viewed as complementary, providing different
insights into the same problem.

Finally, it is important to note that this paper only focuses on the
spatial component of accessibility and does not consider the aspatial
components that also play a significant role in defining an individual's
potential to reach and utilize healthcare services (Joseph and Bantock,
1982). In this regard, future research should utilize the FCA and MNL
approaches for welfare analysis to measure place- and utility-based
accessibility to primary healthcare services for different
socioeconomic, demographic, and mobility profiles. To this end,
place-based accessibility analysis can indirectly consider aspects of
welfare and equity by incorporating factors such as income and mode
share (e.g.~Higgins et al. (2021)). Nevertheless, areal average
accessibilities calculated for different places may not reflect the
subjective preferences, tastes, or needs of heterogeneous individuals or
households. In this regard, the flexibility of random utility-based
approaches can also allow for individual- or household-level
characteristics to be modelled directly as part of tripmaking behavior
and social welfare analyses.

\hypertarget{references}{%
\section*{References}\label{references}}
\addcontentsline{toc}{section}{References}

\hypertarget{refs}{}
\begin{CSLReferences}{1}{0}
\leavevmode\hypertarget{ref-anas1983}{}%
Anas, A., 1983. Discrete choice theory, information theory and the
multinomial logit and gravity models. Transportation Research Part B:
Methodological 17, 13--23.
doi:\href{https://doi.org/10.1016/0191-2615(83)90023-1}{10.1016/0191-2615(83)90023-1}

\leavevmode\hypertarget{ref-apparicio2017}{}%
Apparicio, P., Gelb, J., Dubé, A.-S., Kingham, S., Gauvin, L.,
Robitaille, É., 2017. The approaches to measuring the potential spatial
access to urban health services revisited: distance types and
aggregation-error issues. International Journal of Health Geographics
16.
doi:\href{https://doi.org/10.1186/s12942-017-0105-9}{10.1186/s12942-017-0105-9}

\leavevmode\hypertarget{ref-bauer2016}{}%
Bauer, J., Groneberg, D.A., 2016. Measuring Spatial Accessibility of
Health Care Providers {{}} Introduction of a Variable Distance Decay
Function within the Floating Catchment Area (FCA) Method. PLOS ONE 11,
e0159148.
doi:\href{https://doi.org/10.1371/journal.pone.0159148}{10.1371/journal.pone.0159148}

\leavevmode\hypertarget{ref-ben1985}{}%
Ben-Akiva, M., Lerman, S.R., 1985. Discrete choice analysis: Theory and
application to travel demand. MIT Press.

\leavevmode\hypertarget{ref-brunsdon2020}{}%
Brunsdon, C., Comber, A., 2020. Opening practice: supporting
reproducibility and critical spatial data science. Journal of
Geographical Systems 23, 477--496.
doi:\href{https://doi.org/10.1007/s10109-020-00334-2}{10.1007/s10109-020-00334-2}

\leavevmode\hypertarget{ref-cihi2020}{}%
CIHI, 2020. How {Canada} compares: Results from the commonwealth fund's
2019 international health policy survey of primary care physicians.
Canadian Institute for Health Information.

\leavevmode\hypertarget{ref-dai2010}{}%
Dai, D., 2010. Black residential segregation, disparities in spatial
access to health care facilities, and late-stage breast cancer diagnosis
in metropolitan Detroit. Health \& Place 16, 1038--1052.
doi:\href{https://doi.org/10.1016/j.healthplace.2010.06.012}{10.1016/j.healthplace.2010.06.012}

\leavevmode\hypertarget{ref-dejong2007}{}%
de Jong, G., Daly, A., Pieters, M., van der Hoorn, T., 2007. The logsum
as an evaluation measure: Review of the literature and new results.
Transportation Research Part A: Policy and Practice 41, 874--889.
doi:\href{https://doi.org/10.1016/j.tra.2006.10.002}{10.1016/j.tra.2006.10.002}

\leavevmode\hypertarget{ref-delamater2013}{}%
Delamater, P.L., 2013. Spatial accessibility in suboptimally configured
health care systems: A modified two-step floating catchment area
(M2SFCA) metric. Health \& Place 24, 30--43.
doi:\href{https://doi.org/10.1016/j.healthplace.2013.07.012}{10.1016/j.healthplace.2013.07.012}

\leavevmode\hypertarget{ref-geurs2004}{}%
Geurs, K.T., van Wee, B., 2004. Accessibility evaluation of land-use and
transport strategies: review and research directions. Journal of
Transport Geography 12, 127--140.
doi:\href{https://doi.org/10.1016/j.jtrangeo.2003.10.005}{10.1016/j.jtrangeo.2003.10.005}

\leavevmode\hypertarget{ref-hansen1959}{}%
Hansen, W.G., 1959. How Accessibility Shapes Land Use. Journal of the
American Institute of Planners 25, 73--76.
doi:\href{https://doi.org/10.1080/01944365908978307}{10.1080/01944365908978307}

\leavevmode\hypertarget{ref-hasnine2019}{}%
Hasnine, M.S., Graovac, A., Camargo, F., Habib, K.N., 2019. A random
utility maximization (RUM) based measure of accessibility to transit:
Accurate capturing of the first-mile issue in urban transit. Journal of
Transport Geography 74, 313--320.
doi:\href{https://doi.org/10.1016/j.jtrangeo.2018.12.007}{10.1016/j.jtrangeo.2018.12.007}

\leavevmode\hypertarget{ref-higgins2021}{}%
Higgins, C.D., Páez, A., Kim, G., Wang, J., 2021. Changes in
accessibility to emergency and community food services during COVID-19
and implications for low income populations in Hamilton, Ontario. Social
Science \& Medicine 291, 114442.
doi:\href{https://doi.org/10.1016/j.socscimed.2021.114442}{10.1016/j.socscimed.2021.114442}

\leavevmode\hypertarget{ref-jang2020}{}%
Jang, S., Lee, S., 2020. Study of the regional accessibility calculation
by income class based on utility-based accessibility. Journal of
Transport Geography 84, 102697.
doi:\href{https://doi.org/10.1016/j.jtrangeo.2020.102697}{10.1016/j.jtrangeo.2020.102697}

\leavevmode\hypertarget{ref-joseph1982}{}%
Joseph, A.E., Bantock, P.R., 1982. Measuring potential physical
accessibility to general practitioners in rural areas: A method and case
study. Social Science \& Medicine 16, 85--90.
doi:\href{https://doi.org/10.1016/0277-9536(82)90428-2}{10.1016/0277-9536(82)90428-2}

\leavevmode\hypertarget{ref-kasraian2020}{}%
Kasraian, D., Raghav, S., Miller, E.J., 2020. A multi-decade
longitudinal analysis of transportation and land use co-evolution in the
Greater Toronto-Hamilton Area. Journal of Transport Geography 84,
102696.
doi:\href{https://doi.org/10.1016/j.jtrangeo.2020.102696}{10.1016/j.jtrangeo.2020.102696}

\leavevmode\hypertarget{ref-luo2009}{}%
Luo, W., Qi, Y., 2009. An enhanced two-step floating catchment area
(E2SFCA) method for measuring spatial accessibility to primary care
physicians. Health \& Place 15, 1100--1107.
doi:\href{https://doi.org/10.1016/j.healthplace.2009.06.002}{10.1016/j.healthplace.2009.06.002}

\leavevmode\hypertarget{ref-luo2003}{}%
Luo, W., Wang, F., 2003. Measures of Spatial Accessibility to Health
Care in a GIS Environment: Synthesis and a Case Study in the Chicago
Region. Environment and Planning B: Planning and Design 30, 865--884.
doi:\href{https://doi.org/10.1068/b29120}{10.1068/b29120}

\leavevmode\hypertarget{ref-macfarlane2020}{}%
Macfarlane, G.S., Boyd, N., Taylor, J.E., Watkins, K., 2020. Modeling
the impacts of park access on health outcomes: A utility-based
accessibility approach. Environment and Planning B: Urban Analytics and
City Science 48, 2289--2306.
doi:\href{https://doi.org/10.1177/2399808320974027}{10.1177/2399808320974027}

\leavevmode\hypertarget{ref-mcgrail2009}{}%
McGrail, M.R., Humphreys, J.S., 2009. Measuring spatial accessibility to
primary care in rural areas: Improving the effectiveness of the two-step
floating catchment area method. Applied Geography 29, 533--541.
doi:\href{https://doi.org/10.1016/j.apgeog.2008.12.003}{10.1016/j.apgeog.2008.12.003}

\leavevmode\hypertarget{ref-miller2018}{}%
Miller, E.J., 2018. Accessibility: measurement and application in
transportation planning. Transport Reviews 38, 551--555.
doi:\href{https://doi.org/10.1080/01441647.2018.1492778}{10.1080/01441647.2018.1492778}

\leavevmode\hypertarget{ref-nassir2016}{}%
Nassir, N., Hickman, M., Malekzadeh, A., Irannezhad, E., 2016. A
utility-based travel impedance measure for public transit network
accessibility. Transportation Research Part A: Policy and Practice 88,
26--39.
doi:\href{https://doi.org/10.1016/j.tra.2016.03.007}{10.1016/j.tra.2016.03.007}

\leavevmode\hypertarget{ref-paez2019}{}%
Paez, A., Higgins, C.D., Vivona, S.F., 2019. Demand and level of service
inflation in Floating Catchment Area (FCA) methods. PLOS ONE 14,
e0218773.
doi:\href{https://doi.org/10.1371/journal.pone.0218773}{10.1371/journal.pone.0218773}

\leavevmode\hypertarget{ref-paez2021}{}%
Páez, A., 2021. Open spatial sciences: an introduction. Journal of
Geographical Systems 23, 467--476.
doi:\href{https://doi.org/10.1007/s10109-021-00364-4}{10.1007/s10109-021-00364-4}

\leavevmode\hypertarget{ref-pereira2021}{}%
Pereira, R.H.M., Saraiva, M., Herszenhut, D., Braga, C.K.V., Conway,
M.W., 2021. r5r: Rapid Realistic Routing on Multimodal Transport
Networks with R5 in R. Findings.
doi:\href{https://doi.org/10.32866/001c.21262}{10.32866/001c.21262}

\leavevmode\hypertarget{ref-radke2000}{}%
Radke, J., Mu, L., 2000. Spatial Decompositions, Modeling and Mapping
Service Regions to Predict Access to Social Programs. Annals of GIS 6,
105--112.
doi:\href{https://doi.org/10.1080/10824000009480538}{10.1080/10824000009480538}

\leavevmode\hypertarget{ref-statcan2019}{}%
StatsCan, 2019. Primary health care providers, 2017. Statistics Canada.

\leavevmode\hypertarget{ref-vogel2017}{}%
Vogel, L., 2017. Canadians still waiting for timely access to care.
Canadian Medical Association Journal 189, E375--E376.
doi:\href{https://doi.org/10.1503/cmaj.1095400}{10.1503/cmaj.1095400}

\leavevmode\hypertarget{ref-wan2012}{}%
Wan, N., Zou, B., Sternberg, T., 2012. A three-step floating catchment
area method for analyzing spatial access to health services.
International Journal of Geographical Information Science 26,
1073--1089.
doi:\href{https://doi.org/10.1080/13658816.2011.624987}{10.1080/13658816.2011.624987}

\end{CSLReferences}


\end{document}
