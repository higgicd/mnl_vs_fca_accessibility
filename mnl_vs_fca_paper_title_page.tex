\documentclass[]{elsarticle} %review=doublespace preprint=single 5p=2 column
%%% Begin My package additions %%%%%%%%%%%%%%%%%%%
\usepackage[hyphens]{url}

  \journal{Wellbeing, Space and Society} % Sets Journal name


\usepackage{lineno} % add
  \linenumbers % turns line numbering on
\providecommand{\tightlist}{%
  \setlength{\itemsep}{0pt}\setlength{\parskip}{0pt}}

\usepackage{graphicx}
%%%%%%%%%%%%%%%% end my additions to header

\usepackage[T1]{fontenc}
\usepackage{lmodern}
\usepackage{amssymb,amsmath}
\usepackage{ifxetex,ifluatex}
\usepackage{fixltx2e} % provides \textsubscript
% use upquote if available, for straight quotes in verbatim environments
\IfFileExists{upquote.sty}{\usepackage{upquote}}{}
\ifnum 0\ifxetex 1\fi\ifluatex 1\fi=0 % if pdftex
  \usepackage[utf8]{inputenc}
\else % if luatex or xelatex
  \usepackage{fontspec}
  \ifxetex
    \usepackage{xltxtra,xunicode}
  \fi
  \defaultfontfeatures{Mapping=tex-text,Scale=MatchLowercase}
  \newcommand{\euro}{€}
\fi
% use microtype if available
\IfFileExists{microtype.sty}{\usepackage{microtype}}{}
\bibliographystyle{elsarticle-harv}
\ifxetex
  \usepackage[setpagesize=false, % page size defined by xetex
              unicode=false, % unicode breaks when used with xetex
              xetex]{hyperref}
\else
  \usepackage[unicode=true]{hyperref}
\fi
\hypersetup{breaklinks=true,
            bookmarks=true,
            pdfauthor={},
            pdftitle={Accessibility to Primary Care Physicians: Comparing Floating Catchments with a Utility-based Approach},
            colorlinks=false,
            urlcolor=blue,
            linkcolor=magenta,
            pdfborder={0 0 0}}
\urlstyle{same}  % don't use monospace font for urls

\setcounter{secnumdepth}{0}
% Pandoc toggle for numbering sections (defaults to be off)
\setcounter{secnumdepth}{0}

% Pandoc citation processing

% Pandoc header



\begin{document}
\begin{frontmatter}

  \title{Accessibility to Primary Care Physicians: Comparing Floating
Catchments with a Utility-based Approach}
    \author[University of Toronto]{Maria Demitiry}
   \ead{maria.demitiry@mail.utoronto.ca} 
    \author[University of Toronto Scarborough]{Christopher D.
Higgins\corref{1}}
   \ead{cd.higgins@utoronto.ca} 
    \author[McMaster University]{Antonio Páez}
   \ead{paezha@mcmaster.ca} 
    \author[University of Toronto]{Eric J. Miller}
   \ead{eric.miller@utoronto.ca} 
      \address[University of Toronto]{Department of Civil and Mineral
Engineering, University of Toronto, 35 St.~George Street Toronto, ON.
Canada, M5S 1A4}
    \address[University of Toronto Scarborough]{Department of Human
Geography, 1265 Military Trail, Toronto, ON. Canada, M1C 1A4}
    \address[McMaster University]{School of Earth, Environment and
Society, McMaster University, 1280 Main St W, Hamilton, ON. Canada, L8S
4K1}
      \cortext[1]{Corresponding Author}
  
  \begin{abstract}
  Floating Catchment Area (FCA) methods are a popular choice for
  modelling accessibility to healthcare services because of their
  ability to consider both supply and demand. However, FCA methods do
  not fully consider aspects of travel and choicemaking behaviour as the
  only behavioural component is the impedance function. FCA approaches
  also tend to assign population demand to clinics and levels-of-service
  to population zones in an overlapping manner that has been shown to
  bias results by inflating/deflating supply and demand. While the
  adjustments proposed in the recent ``Balanced FCA'' method can rectify
  this, it apportions population and levels of service in a fractional
  manner. In response, this research proposes a utility-based measure of
  healthcare accessibility based on a multinomial logit (MNL)
  destination choice model that avoids the multiple-counting issue in
  FCA methods and considers several additional behavioural aspects that
  define the appeal of clinics in addition to the travel time required
  to reach them, including their capacity and level of crowding.
  Comparisons of the MNL approach with the original and balanced FCA
  models using data for the City of Hamilton, Canada, suggests that
  while the accessibility patterns produced by each method are broadly
  similar, some key differences exist in the calculated accessibilities
  and their spatial patterns. The MNL model in particular estimates
  higher accessibilities in suburban and rural areas. Based on these
  findings, we argue that both the Balanced FCA and MNL approaches offer
  merit for planning and policy.
  \end{abstract}
   \begin{keyword} healthcare accessibility place-based
accessibility utility-based accessibility destination choice
model accessibility analysis\end{keyword}
 \end{frontmatter}




\end{document}
